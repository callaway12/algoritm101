\documentclass{article}
\usepackage[utf8]{inputenc}
\usepackage{kotex}
\usepackage{graphicx}
\usepackage{amsmath}
\usepackage{mathtools}
\usepackage{commath}


\graphicspath{ {./images/} }

\title{hw2}
\author{B311048 김태와}
\date{2019 03 08}

\begin{document}

\maketitle

\section{1번 과제}
바이너리 서치 트리를 구현, 크고 작은것을 비교하여 애초에 트리 구성을 바로 넣었습니다.
근데 사실 예전에 공부할때 만들었던 트리라 뼈대만 들고와서 원하는 몇번째를 출력하는건 제일 편하게 이미 정렬된 배열을 받았기에 거기서 찾아내게 했습니다.

\section{2번 과제}
퀵소트는 피봇값과 비교해서 작으면 왼쪽 크면 오른쪽해서 그 값들의 배열들을 나눠가며 재귀호출로 피봇만은 자기 위치로 정렬되게끔 했습니다.

\section{3번 과제}
머지 소트는 일단 받은 배열을 계속 반으로 나눠서 2개까지 나누고 그걸 작고 크고를 나눠서 배열에 넣고 그걸 계속 합쳐 나가는걸로 했습니다.

\section{4번 과제}
1번 과제 트리 구조를 그대로 가져오되 입력받는 배열을 굳이 안쓰고 몇개 안되고 원하는 트리의 모습이 나와야하기에 직접 손으로 넣어서 전위 중위 후위 표기를 했습니다.

\end{document}
